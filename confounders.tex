% \documentclass[a4paper,12pt]{extarticle}  % Supports 8pt, 9pt, 10pt, 11pt, 12pt, 14pt, 17pt and 20pt.
\documentclass[12pt]{article}
\setlength{\parindent}{0pt}
\setlength{\parskip}{0.5em} % Adjust '1em' to your preference

% \usepackage[a4paper]{geometry}  % Set the page size to be A4 as default
% \geometry{
%  left=15mm,   % Sets the left margin
%  right=15mm,  % Sets the right margin
%  top=15mm,    % Sets the top margin
%  bottom=15mm  % Sets the bottom margin
% }
\usepackage{geometry}
\geometry{
  left=15mm,   % Sets the left margin
  right=15mm,  % Sets the right margin
  top=15mm,    % Sets the top margin
  bottom=15mm  % Sets the bottom margin
}
\usepackage[utf8]{inputenc}
\usepackage[english]{babel}


\usepackage{graphicx} % Required for inserting images
%In order to put images in the document enter the following command:
%\includegraphics{path/to/image}
%put it in figure environment to add caption and label
%\begin{figure}[H]
%    \centering
%    \includegraphics{path/to/image}
%    \caption{Caption}
%    \label{fig:my_label}
%\end{figure}

%in order to put image alongside text enter the following command:
% \begin{minipage}{0.5\textwidth} % Adjust the width to fit your needs
%   \fbox{\includegraphics[width=\linewidth]{path/to/image}}
% \end{minipage}%
% \hfill % Optional: Adds horizontal space between the minipages
% \begin{minipage}{0.5\textwidth} % Adjust the width accordingly
%   Your text here
% \end{minipage}
%This sill put the image in a box and align it with the text, if you want to add a caption and label you can put the minipage in a figure environment. To remove the box remove the \fbox command.

\usepackage{pdfpages}
% \includepdf[pages=-]{path/to/pdf}


\usepackage{xcolor}

\usepackage{amsthm}
\usepackage{amsfonts} % or \usepackage{amssymb}
\usepackage{amsmath}
\usepackage{mathtools}

\usepackage{hyperref}
\usepackage{float}

% \usepackage{csquotes}  % Recommended for biblatex
% \usepackage[backend=biber, style=numeric]{biblatex}
% \addbibresource{references.bib}  % The filename of the bibliography
\usepackage{natbib}
\bibliographystyle{plainnat}


\usepackage{etoolbox}  % Ensure this package is included

% Define the \citeauthorplus command
\newrobustcmd{\citeauthorplus}[1]{\citeauthor{#1} [\citenum{#1}]}


% \theoremstyle{definition}
\newtheorem{definition}{Definition}
\newtheorem{theorem}{Theorem}
\newtheorem{lemma}{Lemma}
\newtheorem{remark}{Remark}
\newtheorem{example}{Example}
\newtheorem{reminder}{Reminder}
\newtheorem{assumption}{Assumption}
\newtheorem{note}{Note}
\newtheorem{proposition}{Proposition}


\usepackage[noend]{algpseudocode}
\usepackage{algorithm}

\algrenewcommand\algorithmicrequire{\textbf{Input:}}
\algrenewcommand\algorithmicensure{\textbf{Output:}}

\DeclareMathOperator*{\argmin}{arg\,min}
\DeclareMathOperator*{\argmax}{arg\,max}
\DeclarePairedDelimiter\abs{\lvert}{\rvert}%
\DeclarePairedDelimiter\norm{\lVert}{\rVert}%


\usepackage{xcolor}
\newcommand{\naomi}[1]{{\color{red}{Naomi: #1}}}
\newcommand{\gur}[1]{{\color{blue}{Gur: #1}}}
\newcommand{\todo}[1]{{\color{orange}{TODO: #1}}}



\title{The effect of age of enrollment on the probability of graduating in courses - Confounders analysis}
\author{Gur Keinan 213635899, Yarden Adi 212585848}
\date{\today}


\begin{document}

\maketitle

In this work we analyze the effect of the age of enrollment on the probability of graduating in courses. We use the data from the Kaggle competition \href{https://www.kaggle.com/c/2020-ml-final-project}{Predicting Graduation Outcome in Courses}, and was presented in \citet{realinho2021predict}. This dataset contains information about students' demographics, academic performance, and other relevant features. The target variable is the graduation outcome, which can be one of three categories: dropout, enrolled, or graduated. The kaggle competition was about predicting the graduation outcome based on the available features.


We start from the confounders that may affect the probability of graduating in courses.
Following \citet{alyahyan2020predicting}, we identify the following confounders groups:
\begin{itemize}
    \item \textbf{Pre-academic performance}: The student's academic performance before the course. May include high school grades, admission test results, and previous courses grades.
    \item \textbf{Student demographics}: The student's demographic information. May include gender, age, race/ethnicity, socioeconomic status, and family background.
    \item \textbf{Student environment}: The student's environment during the course. May include class type, semester duration, and type of program.
    \item \textbf{Psychological factors}: The student's psychological factors. May include student interest, behavior of study, stress, anxiety, time of preoccupation, self-regulation, and motivation.
\end{itemize}

In addition, we think that Academic progression and Macroeconomic indicators are also important confounders that may affect the probability of graduating in courses.

As for confounders that may affect the age of enrollment, we identify the following groups:
\begin{itemize}
    \item \textbf{Pre-academic performance}: The student's academic performance before the course. May include high school grades, admission test results, and previous courses grades.
    \item \textbf{Student demographics}: The student's demographic information. May include financial status, family background, and cultural background.
    \item \textbf{Student environment}: The student's environment during the course. Can include special programs ("Atuda"), day/night classes, basic / advanced courses.
    \item \textbf{Psychological factors}: The student's psychological factors. May include student interest, behavior of study, stress, anxiety, time of preoccupation, self-regulation, and motivation.
    \item \textbf{External factors}: Factors that are outside the student's control but may significantly influence their age of enrollment. These can include socio-political changes, economic conditions, and shifts in education policy. For instance, economic recessions might delay enrollment as students may need to work before continuing their studies, or educational reforms might alter the standard age of enrollment.
\end{itemize}


After a thorough analysis of the data (while using \href{https://storage.googleapis.com/kaggle-forum-message-attachments/1832313/17922/Features%20information.pdf}{Features information}), we categorize the confounders as follows:

\begin{itemize}
    \item \textbf{Pre-academic performance}:
          \begin{enumerate}
              \item Previous qualification: The level of the student's previous academic qualification, such as secondary education or higher education degrees.
              \item Previous qualification (grade): The grade associated with the student's previous qualification, providing a quantitative measure of past academic performance.
              \item Admission grade: The student's grade upon admission, which could be an indicator of their academic capabilities at the start of the course.
              \item Scholarship holder: Whether the student is a recipient of a scholarship, which might correlate with their academic merit or financial need.
          \end{enumerate}

    \item \textbf{Student demographics}:
          \begin{enumerate}
              \item Gender: The gender of the student.
              \item Marital status: The marital status of the student, which may impact their availability and focus.
              \item Nationality: The nationality of the student, which could be a proxy for cultural and language differences that might influence academic performance.
              \item Age at enrollment: The primary variable of interest, representing the student's age at the time of enrollment.
              \item International: Whether the student is an international student, which may bring additional challenges such as cultural and language barriers.
              \item Mother's qualification: The educational level of the student's mother, which may be a proxy for family support and educational environment.
              \item Father's qualification: The educational level of the student's father, similar to the mother's qualification.
              \item Mother's occupation: The occupation of the student's mother, which might influence the student's socioeconomic status and support.
              \item Father's occupation: The occupation of the student's father, similar to the mother's occupation.
          \end{enumerate}

    \item \textbf{Student environment}:
          \begin{enumerate}
              \item Course: The specific course the student is enrolled in, which may have varying levels of difficulty.
              \item Daytime/evening attendance: Whether the student attends classes during the day or in the evening, which might reflect their time availability and external responsibilities.
              \item Displaced: Whether the student lives away from home during their studies, which could influence their support systems and stress levels.
              \item Educational special needs: Whether the student has any special educational needs that require additional support.
              \item Tuition fees up to date: Whether the student's tuition fees are up to date, which could indicate financial stability.
              \item Debtor: Whether the student has outstanding debts, which could cause stress and affect academic performance.
          \end{enumerate}

    \item \textbf{Psychological factors}: Those are quite complex to measure or to keep track of, one of our weak points.

    \item \textbf{Academic progression}:
          \begin{enumerate}
              \item Curricular units 1st sem (credited): The number of curricular units credited in the 1st semester.
              \item Curricular units 1st sem (enrolled): The number of curricular units the student enrolled in during the 1st semester.
              \item Curricular units 1st sem (evaluations): The number of evaluations the student underwent in the 1st semester.
              \item Curricular units 1st sem (approved): The number of curricular units the student passed in the 1st semester.
              \item Curricular units 1st sem (grade): The average grade of the student in the 1st semester.
              \item Curricular units 1st sem (without evaluations): The number of curricular units the student enrolled in but did not undergo evaluations for in the 1st semester.
              \item Curricular units 2nd sem (credited): The number of curricular units credited in the 2nd semester.
              \item Curricular units 2nd sem (enrolled): The number of curricular units the student enrolled in during the 2nd semester.
              \item Curricular units 2nd sem (evaluations): The number of evaluations the student underwent in the 2nd semester.
              \item Curricular units 2nd sem (approved): The number of curricular units the student passed in the 2nd semester.
              \item Curricular units 2nd sem (grade): The average grade of the student in the 2nd semester.
              \item Curricular units 2nd sem (without evaluations): The number of curricular units the student enrolled in but did not undergo evaluations for in the 2nd semester.
          \end{enumerate}

    \item \textbf{Macroeconomic indicators}:
          \begin{enumerate}
              \item Unemployment rate: The unemployment rate during the period of study, which might impact the student's stress levels and focus.
              \item Inflation rate: The inflation rate during the period of study, which could influence the student's financial stability.
              \item GDP: The Gross Domestic Product during the period of study, reflecting the overall economic environment that may affect the student's circumstances.
          \end{enumerate}
\end{itemize}


This dataset encompasses a wide variety of variables across all relevant categories, allowing for a comprehensive analysis of the factors that influence the probability of graduating in courses. We believe that the extensive coverage of these variables will help mitigate the impact of unknown confounders. Although some potential confounders may remain unobserved, we are confident that they are likely correlated with the variables we already possess, thereby reducing their potential bias in our analysis.

\end{document}





% Dataset “Predict students' dropout and academic success”
% Citation Requests/Acknowledgements
% If you use this dataset in experiments for a scientific publication, please kindly cite our paper
% and/or the dataset available in Zenodo.
% 1. V. Realinho, J. Machado, L. Baptista, M. V. Martins. (2021). “Predict students' dropout and
% academic success” (1.0) [Data set]. Zenodo. DOI: 10.5281/zenodo.5777340
% 2. M. V. Martins, D. Tolledo, J. Machado, L. M. T. Baptista, V. Realinho. (2021) "Early
% prediction of student’s performance in higher education: a case study" Trends and
% Applications in Information Systems and Technologies, vol.1, in Advances in Intelligent
% Systems and Computing series. Springer. DOI: 10.1007/978-3-030-72657-7_16
% Features Information
% Feature Type Description
% Marital status Numeric/discrete 1 – single
% 2 – married
% 3 – widower
% 4 – divorced
% 5 – facto union
% 6 – legally separated
% Application mode Numeric/discrete 1 - 1st phase - general contingent
% 2 - Ordinance No. 612/93
% 5 - 1st phase - special contingent (Azores Island)
% 7 - Holders of other higher courses
% 10 - Ordinance No. 854-B/99
% 15 - International student (bachelor)
% 16 - 1st phase - special contingent (Madeira Island)
% 17 - 2nd phase - general contingent
% 18 - 3rd phase - general contingent
% 26 - Ordinance No. 533-A/99, item b2) (Different Plan)
% 27 - Ordinance No. 533-A/99, item b3 (Other Institution)
% 39 - Over 23 years old
% 42 - Transfer
% 43 - Change of course
% 44 - Technological specialization diploma holders
% 51 - Change of institution/course
% 53 - Short cycle diploma holders
% 57 - Change of institution/course (International)
% Application order Numeric/discrete Application order (between 0 - first choice; and 9 last choice)
% Course Numeric/discrete 33 - Biofuel Production Technologies
% 171 - Animation and Multimedia Design
% 8014 - Social Service (evening attendance)
% 9003 - Agronomy
% 9070 - Communication Design
% 9085 - Veterinary Nursing
% 9119 - Informatics Engineering
% 9130 - Equinculture
% 9147 - Management
% 9238 - Social Service
% 9254 - Tourism
% 9500 - Nursing
% 9556 - Oral Hygiene
% 9670 - Advertising and Marketing Management
% 9773 - Journalism and Communication
% 9853 - Basic Education
% 9991 - Management (evening attendance)
% Daytime/evening
% attendance
% Numeric/binary 1 – daytime
% 0 - evening
% Previous qualification Numeric/discrete 1 - Secondary education
% 2 - Higher education - bachelor's degree
% 3 - Higher education - degree
% 4 - Higher education - master's
% 5 - Higher education - doctorate
% 6 - Frequency of higher education
% 9 - 12th year of schooling - not completed
% 10 - 11th year of schooling - not completed
% 12 - Other - 11th year of schooling
% 14 - 10th year of schooling
% 15 - 10th year of schooling - not completed
% 19 - Basic education 3rd cycle (9th/10th/11th year) or equiv.
% 38 - Basic education 2nd cycle (6th/7th/8th year) or equiv.
% 39 - Technological specialization course
% 40 - Higher education - degree (1st cycle)
% 42 - Professional higher technical course
% 43 - Higher education - master (2nd cycle)
% Previous qualification
% (grade)
% Numeric/continous Grade of previous qualification (between 0 and 200)
% Nacionality Numeric/discrete 1 - Portuguese
% 2 - German
% 6 - Spanish
% 11 - Italian
% 13 - Dutch
% 14 - English
% 17 - Lithuanian
% 21 - Angolan
% 22 - Cape Verdean
% 24 - Guinean
% 25 - Mozambican
% 26 - Santomean
% 32 - Turkish
% 41 - Brazilian
% 62 - Romanian
% 100 - Moldova (Republic of)
% 101 - Mexican
% 103 - Ukrainian
% 105 - Russian
% 108 - Cuban
% 109 - Colombian
% Mother's qualification Numeric/discrete 1 - Secondary Education - 12th Year of Schooling or Eq.
% 2 - Higher Education - Bachelor's Degree
% 3 - Higher Education - Degree
% 4 - Higher Education - Master's
% 5 - Higher Education - Doctorate
% 6 - Frequency of Higher Education
% 9 - 12th Year of Schooling - Not Completed
% 10 - 11th Year of Schooling - Not Completed
% 11 - 7th Year (Old)
% 12 - Other - 11th Year of Schooling
% 14 - 10th Year of Schooling
% 18 - General commerce course
% 19 - Basic Education 3rd Cycle (9th/10th/11th Year) or Equiv.
% 22 - Technical-professional course
% 26 - 7th year of schooling
% 27 - 2nd cycle of the general high school course
% 29 - 9th Year of Schooling - Not Completed
% 30 - 8th year of schooling
% 34 - Unknown
% 35 - Can't read or write
% 36 - Can read without having a 4th year of schooling
% 37 - Basic education 1st cycle (4th/5th year) or equiv.
% 38 - Basic Education 2nd Cycle (6th/7th/8th Year) or Equiv.
% 39 - Technological specialization course
% 40 - Higher education - degree (1st cycle)
% 41 - Specialized higher studies course
% 42 - Professional higher technical course
% 43 - Higher Education - Master (2nd cycle)
% 44 - Higher Education - Doctorate (3rd cycle)
% Father's qualification Numeric/discrete 1 - Secondary Education - 12th Year of Schooling or Eq.
% 2 - Higher Education - Bachelor's Degree
% 3 - Higher Education - Degree
% 4 - Higher Education - Master's
% 5 - Higher Education - Doctorate
% 6 - Frequency of Higher Education
% 9 - 12th Year of Schooling - Not Completed
% 10 - 11th Year of Schooling - Not Completed
% 11 - 7th Year (Old)
% 12 - Other - 11th Year of Schooling
% 13 - 2nd year complementary high school course
% 14 - 10th Year of Schooling
% 18 - General commerce course
% 19 - Basic Education 3rd Cycle (9th/10th/11th Year) or Equiv.
% 20 - Complementary High School Course
% 22 - Technical-professional course
% 25 - Complementary High School Course - not concluded
% 26 - 7th year of schooling
% 27 - 2nd cycle of the general high school course
% 29 - 9th Year of Schooling - Not Completed
% 30 - 8th year of schooling
% 31 - General Course of Administration and Commerce
% 33 - Supplementary Accounting and Administration
% 34 - Unknown
% 35 - Can't read or write
% 36 - Can read without having a 4th year of schooling
% 37 - Basic education 1st cycle (4th/5th year) or equiv.
% 38 - Basic Education 2nd Cycle (6th/7th/8th Year) or Equiv.
% 39 - Technological specialization course
% 40 - Higher education - degree (1st cycle)
% 41 - Specialized higher studies course
% 42 - Professional higher technical course
% 43 - Higher Education - Master (2nd cycle)
% 44 - Higher Education - Doctorate (3rd cycle)
% Mother's occupation Numeric/discrete 0 - Student
% 1 - Representatives of the Legislative Power and Executive Bodies, Directors,
% Directors and Executive Managers
% 2 - Specialists in Intellectual and Scientific Activities
% 3 - Intermediate Level Technicians and Professions
% 4 - Administrative staff
% 5 - Personal Services, Security and Safety Workers and Sellers
% 6 - Farmers and Skilled Workers in Agriculture, Fisheries and Forestry
% 7 - Skilled Workers in Industry, Construction and Craftsmen
% 8 - Installation and Machine Operators and Assembly Workers
% 9 - Unskilled Workers
% 10 - Armed Forces Professions
% 90 - Other Situation
% 99 - (blank)
% 122 - Health professionals
% 123 - teachers
% 125 - Specialists in information and communication technologies (ICT)
% 131 - Intermediate level science and engineering technicians and professions
% 132 - Technicians and professionals, of intermediate level of health
% 134 - Intermediate level technicians from legal, social, sports, cultural and
% similar services
% 141 - Office workers, secretaries in general and data processing operators
% 143 - Data, accounting, statistical, financial services and registry-related
% operators
% 144 - Other administrative support staff
% 151 - personal service workers
% 152 - sellers
% 153 - Personal care workers and the like
% 171 - Skilled construction workers and the like, except electricians
% 173 - Skilled workers in printing, precision instrument manufacturing,
% jewelers, artisans and the like
% 175 - Workers in food processing, woodworking, clothing and other industries
% and crafts
% 191 - cleaning workers
% 192 - Unskilled workers in agriculture, animal production, fisheries and
% forestry
% 193 - Unskilled workers in extractive industry, construction, manufacturing
% and transport
% 194 - Meal preparation assistants
% Father's occupation Numeric/discrete 0 - Student
% 1 - Representatives of the Legislative Power and Executive Bodies, Directors,
% Directors and Executive Managers
% 2 - Specialists in Intellectual and Scientific Activities
% 3 - Intermediate Level Technicians and Professions
% 4 - Administrative staff
% 5 - Personal Services, Security and Safety Workers and Sellers
% 6 - Farmers and Skilled Workers in Agriculture, Fisheries and Forestry
% 7 - Skilled Workers in Industry, Construction and Craftsmen
% 8 - Installation and Machine Operators and Assembly Workers
% 9 - Unskilled Workers
% 10 - Armed Forces Professions
% 90 - Other Situation
% 99 - (blank)
% 101 - Armed Forces Officers
% 102 - Armed Forces Sergeants
% 103 - Other Armed Forces personnel
% 112 - Directors of administrative and commercial services
% 114 - Hotel, catering, trade and other services directors
% 121 - Specialists in the physical sciences, mathematics, engineering and
% related techniques
% 122 - Health professionals
% 123 - teachers
% 124 - Specialists in finance, accounting, administrative organization, public
% and commercial relations
% 131 - Intermediate level science and engineering technicians and professions
% 132 - Technicians and professionals, of intermediate level of health
% 134 - Intermediate level technicians from legal, social, sports, cultural and
% similar services
% 135 - Information and communication technology technicians
% 141 - Office workers, secretaries in general and data processing operators
% 143 - Data, accounting, statistical, financial services and registry-related
% operators
% 144 - Other administrative support staff
% 151 - personal service workers
% 152 - sellers
% 153 - Personal care workers and the like
% 154 - Protection and security services personnel
% 161 - Market-oriented farmers and skilled agricultural and animal production
% workers
% 163 - Farmers, livestock keepers, fishermen, hunters and gatherers,
% subsistence
% 171 - Skilled construction workers and the like, except electricians
% 172 - Skilled workers in metallurgy, metalworking and similar
% 174 - Skilled workers in electricity and electronics
% 175 - Workers in food processing, woodworking, clothing and other industries
% and crafts
% 181 - Fixed plant and machine operators
% 182 - assembly workers
% 183 - Vehicle drivers and mobile equipment operators
% 192 - Unskilled workers in agriculture, animal production, fisheries and
% forestry
% 193 - Unskilled workers in extractive industry, construction, manufacturing
% and transport
% 194 - Meal preparation assistants
% 195 - Street vendors (except food) and street service providers
% Admission grade Numeric/continous Admission grade (between 0 and 200)
% Displaced Numeric/binary 1 – yes
% 0 – no
% Educational special
% needs
% Numeric/binary 1 – yes
% 0 – no
% Debtor Numeric/binary 1 – yes
% 0 – no
% Tuition fees up to date Numeric/binary 1 – yes
% 0 – no
% Gender Numeric/binary 1 – male
% 0 – female
% Scholarship holder Numeric/binary 1 – yes
% 0 – no
% Age at enrollment Numeric/discrete Age of studend at enrollment
% International Numeric/binary 1 – yes
% 0 – no
% Curricular units 1st sem
% (credited)
% Numeric/discrete Number of curricular units credited in the 1st semester
% Curricular units 1st sem
% (enrolled)
% Numeric/discrete Number of curricular units enrolled in the 1st semester
% Curricular units 1st sem
% (evaluations)
% Numeric/discrete Number of evaluations to curricular units in the 1st semester
% Curricular units 1st sem
% (approved)
% Numeric/discrete Number of curricular units approved in the 1st semester
% Curricular units 1st sem
% (grade)
% Numeric/discrete Grade average in the 1st semester (between 0 and 20)
% Curricular units 1st sem
% (without evaluations)
% Numeric/discrete Number of curricular units without evalutions in the 1st semester
% Curricular units 2nd
% sem (credited)
% Numeric/discrete Number of curricular units credited in the 2nd semester
% Curricular units 2nd
% sem (enrolled)
% Numeric/discrete Number of curricular units enrolled in the 2nd semester
% Curricular units 2nd
% sem (evaluations)
% Numeric/discrete Number of evaluations to curricular units in the 2nd semester
% Curricular units 2nd
% sem (approved)
% Numeric/discrete Number of curricular units approved in the 2nd semester
% Curricular units 2nd
% sem (grade)
% Numeric/discrete Grade average in the 2nd semester (between 0 and 20)
% Curricular units 2nd
% sem (without
% evaluations)
% Numeric/discrete Number of curricular units without evalutions in the 1st semester
% Unemployment rate Numeric/continous Unemployment rate (%)
% Inflation rate Numeric/continous Inflation rate (%)
% GDP Numeric/continous GDP
% Target Categorical Target. The problem is formulated as a three category classification task
% (dropout, enrolled, and graduate) at the end of the normal duration of the
% course.

\bibliography{references}


\end{document}

